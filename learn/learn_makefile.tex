%
% file : learn_git.tex
% date : jeudi 19 décembre 2019, 13:19:09 (UTC+0100)
% author : sedelpeuch
% description :
\documentclass[a4paper,10pt]{article}
\usepackage[utf8]{inputenc}
\usepackage[T1]{fontenc}
\usepackage[french]{babel}
\usepackage{graphicx}
\usepackage{float}
\usepackage{amsmath}
\usepackage{amssymb}
\usepackage{mathrsfs}
\usepackage{color}
\usepackage{fancyhdr}
\usepackage{pdfpages}
\usepackage{layout}
\usepackage{multicol}
\usepackage{setspace}
\usepackage{csvsimple}
\usepackage[table]{xcolor}
\usepackage[colorlinks=true]{hyperref}
\usepackage{tikz, tkz-tab}
\usepackage[top=2cm,bottom=2cm,left=2cm,right=2cm]{geometry}
\usepackage{amsthm}
\usepackage{listings}
\setlength{\parindent}{0cm}
\setlength{\parskip}{1ex plus 0.5ex minus 0.2ex}
\newcommand{\hsp}{\hspace{20pt}}
\newcommand{\HRule}{\rule{\linewidth}{0.1mm}}

\definecolor{darkWhite}{rgb}{0.94,0.94,0.94}

\lstset{
 aboveskip=3mm,
  belowskip=-2mm,
  backgroundcolor=\color{white},
  basicstyle=\footnotesize,
  breakatwhitespace=false,
  breaklines=true,
  captionpos=b,
  commentstyle=\color{red},
  deletekeywords={...},
  escapeinside={\%*}{(*)},
  extendedchars=true,
  framexleftmargin=16pt,
  framextopmargin=3pt,
  framexbottommargin=6pt,
  frame=tb,
  keepspaces=true,
  keywordstyle=\color{blue},
  language=bash,
  literate=
  {²}{{\textsuperscript{2}}}1
  {⁴}{{\textsuperscript{4}}}1
  {⁶}{{\textsuperscript{6}}}1
  {⁸}{{\textsuperscript{8}}}1
  {€}{{\euro{}}}1
  {é}{{\'e}}1
  {è}{{\`{e}}}1
  {ê}{{\^{e}}}1
  {ë}{{\¨{e}}}1
  {É}{{\'{E}}}1
  {Ê}{{\^{E}}}1
  {û}{{\^{u}}}1
  {ù}{{\`{u}}}1
  {â}{{\^{a}}}1
  {à}{{\`{a}}}1
  {á}{{\'{a}}}1
  {ã}{{\~{a}}}1
  {Á}{{\'{A}}}1
  {Â}{{\^{A}}}1
  {Ã}{{\~{A}}}1
  {ç}{{\c{c}}}1
  {Ç}{{\c{C}}}1
  {õ}{{\~{o}}}1
  {ó}{{\'{o}}}1
  {ô}{{\^{o}}}1
  {Õ}{{\~{O}}}1
  {Ó}{{\'{O}}}1
  {Ô}{{\^{O}}}1
  {î}{{\^{i}}}1
  {Î}{{\^{I}}}1
  {í}{{\'{i}}}1
  {Í}{{\~{Í}}}1,
  morekeywords={*,...},
  numbers=left,
  numbersep=10pt,
  numberstyle=\tiny\color{black},
  rulecolor=\color{black},
  showspaces=false,
  showstringspaces=false,
  showtabs=false,
  stepnumber=1,
  stringstyle=\color{gray},
  tabsize=4,
  title=\lstname,
}
\begin{document}
\begin{spacing}{1.5}
\graphicspath{{image/}}
\setcounter{tocdepth}{2}
\newpage
\pagestyle{fancy} \lhead{} \chead{\textbf{Guide de survie Makefile}}
\rhead{\thepage} \lfoot{} \cfoot{} \fancyfoot[R] {
  \includegraphics[scale=0.075]{65508.png} }
\HRule
\begin{center}
  \LARGE \textbf{Guide de survie pour Makefile}
\end{center}
\HRule \\

\subsection*{Un Makefile c'est quoi ?}
Les Makefiles sont des fichiers permettant d'executer un ensemble d'actions,
comme la compilation d'un projet, la mise à jour d'un rapport, où l'archivage de
données. Les Makefiles sont des fichiers shell particulier ils respectent
cependant les conventions de codages du shell. L'idée de
ce document est de présenté l'idée générale d'un Makefile via un exemple
basique, puis d'aller vers un exemple plus complexe, le Makefile pour notre
robot.

\subsection*{Un exemple basique}
Considérons un exemple de projet très basique, constitué de 3 fichiers :
hello.c, hello.h et main.c. Nous pouvons faire le graphe des dépendances
suivants
\begin{center}
\begin{tikzpicture}
\tikzstyle{lien}=[-,>=stealth,rounded corners=20pt,thick]
\tikzset{individu/.style={draw,thick,fill=#1!25},
individu/.default={white}}
\node[individu=blue] (A) at (0,0 ){main.c};
\node[individu] (A1) at (2,2){hello.c};
\node[individu] (A2) at (2,0){hello.h};


\draw[lien] (A2) -- (A);
\draw[lien] (A1) -- (A2);
\end{tikzpicture}
\end{center}
De manière plus littérale nous avons définit des fonctions dans hello.c, ces
fonctions sont reférencées par leurs prototypes dans hello.h. Par ailleurs nous
avons définies des fonctions dans main.c, nous sous entendons que les fonctions
de main.c font appels à celles de hello.c (d'où l'inclusion de hello.h dans
main.c). \\

Nous voulons donc compiler main.c pour créer un executable que nous appellerons
\textit{project}. Si nous essayons directement de faire ``gcc -Wall -Werror
-std=c99 main.c project'' nous allons avoir des références indéfinies. En effet
il faut d'abbord compiler hello.c en hello.o pour que la machine connaisse les
fonctions définit dans le hello.c (c'est le rôle du hello.o). Ainsi pour créer
project il faut faire deux lignes de commande ``gcc -Wall -Werror -std=c99
hello.c -c'' puis ``gcc -Wall -Werror -std=c99 hello.o main.c -o project''. \\

Deux nouvelles options se présentent ``-c'' qui permet de dire au compilateur de
créer un fichier ``.o'' et ``-o'' qui permet d'appeler le linker (c'est lui qui
dit au compilateur que certainnes fonctions appelées dans main.c sont définies
dans hello.o).\\

Maintenant que nous avons compris comment nous pouvions compiler ce projet, nous
allons regarder comment automatiser ces étapes. Pour cela regardons la synthaxe
type d'un Makefile.

\begin{lstlisting}
  cible : dependances
          commandes
\end{lstlisting}

Analysons ces deux lignes de commandes, le ``cible'' permet de donner un nom
ainsi dans le shell nous pourrons taper ``make cible'' pour effectuer les
actions correspondantes. Dans notre cas la cible serait ``project''. Ensuite les
``dépendances'' sont les fichiers nécessaires à la création de la cible, dans
notre cas cela serait ``hello.o''. Un exemple de

\begin{lstlisting}
  cible : dependances
          commandes
\end{lstlisting}

\newpage
\end{spacing}
\end{document}
